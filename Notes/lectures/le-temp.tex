\documentclass[../main.tex]{subfiles}
\usepackage{csquotes}
\usepackage{fontspec}
\setmonofont{FreeMono}
\begin{document}

\section{Lecture 2}
Today: The structure theorem and modules over a PID.
\subsection{Modules Over a PID}
\begin{definition}
	A commutative ring $R$ with unity ($1 \neq 0$) is a \textbf{principal ideal domain (PID)} if
	\begin{itemize}
		\item $R$ is an integral domain (no zero divisors), and
		\item every ideal of $R$ is principal, i.e. for all $I \subseteq R$ there exists $a \in R$ such that $I = (a)$.
	\end{itemize}
\end{definition}
\begin{example} Examples of PIDs:
	\begin{itemize}
		\item $\mathbb{Z}$ is a PID.
		\item $k[x]$ is a PID, where $k$ is a field.
	\end{itemize}
\end{example}
\begin{proposition}
	Let $M$ be a module over a PID $R$. Then
	\[
		M \text{ is injective } \iff M \text{ is divisible},
	\]
	i.e. for all $m \in M$ and all $r \in R \setminus \{0\}$ there exists $m' \in M$ such that $m = r m'$.
\end{proposition}
\begin{proof}
	($\Rightarrow$) Suppose $M$ is injective. Let $r \in R \setminus \{0\}$ and $m \in M$. Define $f:(r) \to M$ by $f(r)=m$. Since $M$ is injective, the inclusion $(r) \hookrightarrow R$ extends to $g:R \to M$ with $g\vert_{(r)}=f$ and the diagram
	\[
		\begin{tikzcd}
			I=(r) \arrow[r, "f"] \arrow[d, hook] & M \\
			R \arrow[ru, dashed, "\exists g"]
		\end{tikzcd}
	\]
	commutes. Then $m=f(r)=g(r)=r g(1)$, so $m$ is divisible by $r$.

	($\Leftarrow$) Suppose $M$ is divisible. Let $I \subseteq R$ be an ideal and $f:I \to M$ an $R$-module homomorphism. Since $R$ is a PID, $I=(a)$ for some $a \in R$. If $a=0$, then $I=0$ and the zero map extends. If $a \neq 0$, divisibility gives $m' \in M$ with $f(a)=a m'$. Define $g:R \to M$ by $g(r)=r m'$. Then $g$ is an $R$-module homomorphism and $g(a)=a m'=f(a)$, so $g\vert_{(a)}=f$. Hence $M$ is injective.
\end{proof}
\end{document}
